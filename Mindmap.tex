\documentclass[cn, hazy, blue!60, normal, 14pt]{elegantnote}

\title{数字信号处理各种域与变换的关系}
\author{Xiaohei}
\version{1.2}
\date{\zhtoday}

\usepackage{tikz}

\tikzset{
    time/.style ={
        circle,
        minimum width   = 60pt,
        minimum height  = 60pt,
        inner sep       = 5pt,
        draw            = teal
    },
    freq/.style ={
        circle,
        minimum width   = 60pt,
        minimum height  = 60pt,
        inner sep       = 5pt,
        draw            = orange
    },
    cfrq/.style ={
        circle,
        minimum width   = 60pt,
        minimum height  = 60pt,
        inner sep       = 5pt,
        draw            = cyan
    }
}

\linespread{1.0}

\begin{document}

\maketitle

\begin{center}
\begin{tikzpicture}[ultra thick]
    % 1 离散周期时域
    \node[time] (1) at(0,0) {$\widetilde{x}(n)$};

    % 2 离散周期频域
    \node[freq] (2) at(10,0) {$\widetilde{X}(k)$};

    % 3 离散时域
    \node[time] (3) at(0,6) {$x(n)$};

    % 4 离散频域
    \node[freq] (4) at(5,3) {$X(k)$};

    % 5 数字频域
    \node[freq] (5) at(10,6) {$X\left(\text{e}^{\text{j}\omega}\right)$};

    % 6 z复频域
    \node[cfrq] (6) at(5,9) {$X(z)$};

    % 7 模拟时域
    \node[time] (7) at(0,12) {$x(t)$};

    % 8 模拟频域
    \node[freq] (8) at(10,12) {$X\left(\text{j}\Omega\right)$};

    % 9 s复频域
    \node[cfrq] (9) at(5,15) {$X(s)$};

    % 1 2
    \draw[{latex[red!60]}-{latex[blue!60]}] (1) -- (2);
    \node[above, blue!60] at(5,0) {DFS};
    \node[below, red!60] at(5,0) {IDFS};

    % 1 3
    \draw[{latex[red!60]}-{latex[blue!60]}] (1) -- (3);
    \node[left, blue!60, text width=1em] at(0,3) {取主值序列};
    \node[right, red!60, text width=1em] at(0,3) {周期延拓};

    % 2 4
    \draw[{latex[red!60]}-{latex[blue!60]}] (2) -- (4);
    \node[left, blue!60] at(7.5,1.2) {取主值周期};
    \node[right, red!60] at(7.5,1.8) {周期延拓};

    % 3 4
    \draw[{latex[red!60]}-{latex[blue!60]}] (3) -- (4);
    \node[left, blue!60] at(2.5,4.2) {DFT};
    \node[right, red!60] at(2.5,4.8) {IDFT};
    
    % 2 5
    \draw[{latex[red!60]}-{latex[blue!60]}] (2) -- (5);
    \node[left, blue!60, text width=1em] at(10,3) {插值};
    \node[right, red!60, text width=1em] at(10,3) {采样};

    % 4 5
    \draw[{latex[red!60]}-{latex[blue!60]}] (4) -- (5);
    \node[left, blue!60] at(7.5,4.8) {插值};
    \node[right, red!60] at(7.5,4.2) {采样};

    % 3 6
    \draw[{latex[red!60]}-{latex[blue!60]}] (3) -- (6);
    \node[left, blue!60] at(2.5,7.8) {ZT};
    \node[right, red!60] at(2.5,7.2) {IZT};

    % 4 6
    \draw[{latex[red!60]}-{latex[blue!60]}] (4) -- (6);
    \node[left, blue!60, text width=1em] at(5,6) {插值};
    \node[right, red!60] at(5,6) {$z=W_{n}^{-k}$};

    % 5 6
    \draw[{latex[red!60]}-] (5) -- (6);
    \node[right, red!60] at(7,8) {$z=\text{e}^{\text{j}\omega}$};

    % 3 7
    \draw[{latex[red!60]}-{latex[blue!60]}] (3) -- (7);
    \node[left, blue!60, text width=1em] at(0,9) {插值};
    \node[right, red!60, text width=1em] at(0,9) {采样};

    % 5 8
    \draw[{latex[red!60]}-{latex[blue!60]}] (5) -- (8);
    \node[left, blue!60, text width=1em] at(10,9) {取主值周期};
    \node[right, red!60, text width=1em] at(10,9) {归一化延拓};

    % 7 8
    \draw[{latex[red!60]}-{latex[blue!60]}] (7) -- (8);
    \node[above, blue!60] at(7,12) {FT};
    \node[below, red!60] at(7,12) {IFT};

    % 6 9
    \draw[{latex[red!60]}-] (6) -- (9);
    \node[left, red!60] at(5,11) {$\text{e}^{sT}=z$};

    % 7 9
    \draw[{latex[red!60]}-{latex[blue!60]}] (7) -- (9);
    \node[left, blue!60] at(2.5,13.8) {LT};
    \node[right, red!60] at(2.5,13.2) {ILT};

    % 8 9
    \draw[{latex[red!60]}-{latex[blue!60]}] (8) -- (9);
    \node[left, blue!60] at(7.5,13.2) {$s=\text{j}\Omega$};
    \node[right, red!60] at(7.5,13.8) {解析开拓};

\end{tikzpicture}
\end{center}

\end{document}
